People are beginning to realise that programming is a fundamental skill in our
increasingly computerised society.  Not just for software engineers and
computer scientists, but for everyone. However, the present languages and tools
used to teach programming have several problems. They often have bad error
messages filled with concepts that are beyond beginners, and they suffer from
having to write ``magic incantations''.  The tools sometimes have little or bad
support on some platforms, creating a barrier to entry for inexperienced
students.

We propose to solve these issues by making a programming language, based on
Haskell, for teaching, with unlockable levels of features. This solves the 
first two described problems. To make the barrier to entry as low as possible 
we propose the engineering of an online IDE where users may write and run code 
in the browser. This IDE also opens up many other exciting possibilities, such 
as collaboration, sharing, virtual classrooms with integrated exercises, and 
game programming.

The existing tools we have analysed have unfortunate shortcomings. Our 
solution on the other hand is well cool.
