\subsection{Language} \label{sec:Language}

We want to make an implementation of the Haskell programming
language~\cite{marlow2010haskell} with unlockable sets of features.  Haskell is
a declarative programming language suitable and often preferred for use as the
first taught language in computer science courses~\cite{dijkstra2001members},
and is regarded as an elegant programming language that allows writing programs
in many paradigms including functional and imperative.

With unlockable features we mean that a student may start out with a language
with very few features turned on. As a result, we can provide very relevant
error messages tailored to their level of expertise and that do not mention
features that they have not yet encountered. As the teaching progresses, we can
turn on increasingly advanced features of the language.  Since we are using an
already established, general-purpose language it means that the knowledge
gained using our implementation transfers to ordinary implementations of
Haskell, and by having unlockable features we avoid the problems associated
with using a general-purpose language for teaching.

\subsection{Integrated development environment} \label{sec:IDE}

To solve the problem of installation difficulties, we propose to provide an
online IDE for our language.  This means that a student can simply point their
web browser to a website providing the IDE and immediately start writing and
running code in the browser; the barrier to entry is extremely low. It also
means that we can provide the same up-to-date version of the language and its
libraries to everyone.

An online IDE opens the door to many features that will help the students to
learn or just make programming exciting:
\begin{description}
  \item[Collaborating]
    Students can form groups and collaboratively write code.  Since the IDE is
    online they can do this even if they are not physically in the same
    location, and they can edit the same file simultaneously.
  \item[Sharing]
    Students can easily share their code and applications with friends and
    classmates.
  \item[Writing games] 
    We can provide libraries to write games that run in the browser, and
    provide easy access to art assets in the IDE.
  \item[Learning]
    Teachers can provide learning material with exercises that students can do
    and have automatically marked directly in the IDE. Language features (or
    even game assets) can be unlocked as the students progress through the
    exercises.
\end{description}

\subsection{Academic use}

We have already touched on a few uses of our language and IDE in the academic
sector, but there are many more possibilities.
Our IDE could be specialised for programming courses, with virtual classrooms
where students and tutors could meet to discuss the course and its contents.
Teachers could author their own custom curriculums and exercises and we could
integrate the IDE with their learning management system.

By providing different libraries and toggling different sets of features of the
language it could be customised to suit many diverse kinds of courses at
different levels in the education system, be it programming fundamentals at the
university or game development for kids.

\subsection{Commercialisation}

We would like to keep the basic IDE gratis for everyone, but we still have
several ideas that make the idea commercially viable:

\begin{description}
  \item[Premium features]
    For instance the classroom features could be a premium feature that schools
    and universities would have to pay for.
  \item[Art assets]
    Art assets for use in games and applications could be sold directly in the
    IDE.
  \item[Application store]
    Users could be allowed to publish and sell their creations in an 
    associated application store in return for a portion of their revenue.
\end{description}

\subsection{Societal benefits}
Whilst user interface design is constantly improving, there is no denying the 
inherent complexity of a modern day computer. In a society where only software 
developers understand the basic science of a computer program, unenlightened 
computer users are left helpless. This is problematic in the case of 
proprietary software, where computer users are slaves to the subjugation of 
the power elite created by these software developers -- and worst of all, they 
often don't even realise it. Learning to program a computer will as a 
side-effect make computer users more aware of how computers actually work, 
which in turn gives them the foundation for self-educating themselves further. 
Furthermore, understanding the basics of computer programming means 
understanding what source code is, and why it must be free for computer users 
themselves to be free. Illuminating users in this manner is among the zenith 
of societal contribution in computer science.

Another aspect to consider is that teaching computer programming in a 
principled way is The Right Thing. Computer programmers today are typically 
either lone self-taught hackers, or computer science students. Both of these 
groups suffer setbacks from learning to program through pedagogically unsound 
methodologies. Others never make it to an enlightened state at all. Society at 
large is starting to take this seriously. Computing at 
School~\footnote{\url{http://www.computingatschool.org.uk/}} has been 
successful in improving national school curricula in the UK, and in Norway we 
have Lær Kidsa Koding\footnote{\url{http://www.kidsakoder.no/}}.

\subsection{Timetable}
