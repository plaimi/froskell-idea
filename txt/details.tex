\subsection{Language}

We want to make an implementation of the Haskell programming
language~\cite{marlow2010haskell} with unlockable sets of features.  Haskell is
a declarative programming language suitable and often preferred for use as the
first taught language in computer science courses~\cite{dijkstra2001members},
and is regarded as an elegant programming language that allows writing programs
in many paradigms including functional and imperative.

With unlockable features we mean that a student may start out with a language
with very few features turned on. As a result, we can provide very relevant
error messages tailored to their level of expertise and that do not mention
features that they have not yet encountered. As the teaching progresses, we can
turn on increasingly advanced features of the language.  Since we are using an
already established, general-purpose language it means that the knowledge
gained using our implementation transfers to ordinary implementations of
Haskell, and by by having unlockable features we avoid the problems associated
with using a general-purpose language for teaching.

\subsection{Integrated development environment}

\subsection{Academic use}

\subsection{Commercialisation}

\subsection{Societal benefits}

\subsection{Timetable}
