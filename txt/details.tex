\subsection{Language}

We want to make an implementation of the Haskell programming
language~\cite{marlow2010haskell} with unlockable sets of features.  Haskell is
a declarative programming language suitable and often preferred for use as the
first taught language in computer science courses~\cite{dijkstra2001members},
and is regarded as an elegant programming language that allows writing programs
in many paradigms including functional and imperative.

With unlockable features we mean that a student may start out with a language
with very few features turned on. As a result, we can provide very relevant
error messages tailored to their level of expertise and that do not mention
features that they have not yet encountered. As the teaching progresses, we can
turn on increasingly advanced features of the language.  Since we are using an
already established, general-purpose language it means that the knowledge
gained using our implementation transfers to ordinary implementations of
Haskell, and by having unlockable features we avoid the problems associated
with using a general-purpose language for teaching.

\subsection{Integrated development environment}

To solve the problem of installation difficulties, we propose to provide an
online \emph{integrated development environment}~(IDE) for our language.  This
means that a student can simply point their web browser to a website providing
the IDE and immediately start writing and running code in the browser; the
barrier to entry is extremely low. It also means that we can provide the same
up-to-date version of the language and its libraries to everyone.

An online IDE opens the door to many features that will help the students to
learn or just make programming exciting:
\begin{description}
  \item[Collaborating]
    Students can form groups and collaboratively write code.  Since the IDE is
    online they can do this even if they are not physically in the same
    location, and they can edit the same file simultaneously.
  \item[Sharing]
    Students can easily share their code and applications with friends and
    classmates.
  \item[Writing games] 
    We can provide libraries to write games that run in the browser, and
    provide easy access to art assets in the IDE.
  \item[Learning]
    Teachers can provide learning material with exercises that students can do
    and have automatically marked directly in the IDE. Language features (or
    even game assets) can be unlocked as the students progress through the
    exercises.
\end{description}

\subsection{Academic use}

\subsection{Commercialisation}

\subsection{Societal benefits}

\subsection{Timetable}
