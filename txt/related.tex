\subsection{Historical languages}
In this section we look at some existing solutions to the described problem, 
and why they are not ideal. Let's start with some of the most popular 
historical languages.

BASIC was authored to provide a programming language which would be easy to 
learn for students without a rigorous mathematical background. It came about 
in the '60s, designed for use with Darthmouth's timesharing system, and became 
truly influential during the home computer revolution of the '70s. There have 
been several versions of BASIC since the original; the most notable dialect 
arguably being Microsoft's Visual Basic\cite{time2014basic}.

Pascal was designed in 1971, partly as a simplified version of Algol, partly 
as a language that encouraged structured programming. It was designed for 
educational purposes, but evolved into a popular general-purpose programming 
language\cite{cantu2008essential}. The initial versions were criticised for 
not being suitable for proper programming\cite{kernighan1981pascal}. Several 
versions and dialects of Pascal have since emerged\cite{cantu2008essential}, 
that eliminate these problems.

Scheme is a Lisp programming language that was originally designed with 
tutorial purposes in mind\cite{sussman1998scheme}. It was used in the 
influential book Structure and Interpretation of Computer Programs, which was 
used at MIT to teach programming. The success of the book and the language 
itself later lead to Scheme becoming a popular choice for introductory 
programming courses at other universities too\cite{felleisen2004structure}.

Logo is another Lisp programming language, made specifically for teaching 
programming to children. A notable part of several Logo environments is the 
use of a turtle avatar that moves around the screen and draws things. It is an 
influential language which has seen a lot of adaptation since its 
inception\cite{logo2011whatislogo}.

None of the languages mentioned here solve any of the problems we have 
identified. Furthermore, they are antiquated.

\subsection{Helium}

\subsection{Alice}

\subsection{Oz/Mozart}

\subsection{Online interactive IDEs not necessarily aimed at teaching}
