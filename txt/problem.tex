When teaching programming languages, the choice of language is important.  Some
teachers opt for using languages tailor-made for teaching. These generally fall
into two categories: those that are a simplified version of a general-purpose
language, and those that are completely new languages. One benefit of both
language categories is that their error messages can be better, for instance by
never containing references to advanced features that the student has not yet
been taught. As an example, a Haskell student that writes
\begin{verbatim}
x = abs -3
\end{verbatim}
is met with a long error message that contains the following snippets:
\begin{verbatim}
No instance for (Num a0) arising from
  a use of ‘abs’
The type variable ‘a0’ is ambiguous
\end{verbatim}
and
\begin{verbatim}
No instance for (Num (a0 -> a0)) arising
  from a use of ‘-’
In the expression: abs - 3
\end{verbatim}
The actual error is that the snippet is interpreted as \texttt{abs - 3}, i.e.
that the minus sign is taken as subtraction of \texttt{3} from \texttt{abs} and
not as negation.  However, the error message speaks of \texttt{Num} typeclass
instances and type variables -- concepts that a beginner Haskell programmer is
almost guaranteed to not know.

When using a teaching language we can also avoid ``magic incantations''. As an
example, take the following Java program that prints ``Hello world!'' to the
console:
\begin{verbatim}
public class HelloWorld {
  public static void
  main(String[] args) {
    System.out.println("Hello world!");
  }
}
\end{verbatim}
To teach what this program does the teacher has to either tell the student to
not worry about the first three lines for now, probably leaving the student in
a puzzled state, or hold a long lecture about concepts not entirely relevant to
a beginner.

Another difficulty in teaching programming is that it is often complicated for
students to install the programming environment on their own machines. This
occurs for several reasons, for instance because the language might have bad or
no support for certain operating systems or simply because the student does not
yet have enough computer experience. It can also be difficult to provide
consistent versions of the language implementation and any required libraries.
These difficulties lead to the teacher having to provide what amounts to technical
instead of being able to focus only on programming.
